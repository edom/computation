\chapter{Pricing}

\section{Non-risky instruments}

Time preference.
Discount.

The price of a risky instrument
is the risk-neutral price plus the \emph{risk premium}.

\chapter{Risk-neutral price of risky instruments}

Let $X$ be random process of the price.
$X t$ is the random variable for the price at time $t$.

\section{Futures contracts}

The risk-neutral price of futures contract with $T$ time to expiration.
\begin{align}
    \int_{\mathbb{R}} p [X T = x] \cdot x \cdot e^{- r \cdot T} \cdot dx
\end{align}

\section{European call options}

The risk-neutral price of European call option
that has strike price $k$ and expires in time $T$ from now
should be
\begin{align}
    \int_k^\infty p [X T = x] \cdot (x - k) \cdot e^{- r \cdot T} \cdot dx
\end{align}
or, discretely,
\begin{align}
    \sum_{x \in \mathbb{X}} p[X T = x] \cdot (x - k) \cdot e^{- r \cdot T} \cdot \Delta x,
\end{align}
possibly with $\Delta x = \$ 0.01$
and $p$ taken from binomial distribution.

Given any two of these three quantities,
the other one can be computed:
\begin{itemize}
    \item actual current price,
    \item implied distribution of price at expiration,
    \item risk-free interest rate.
\end{itemize}

\section{American call options}

The risk-neutral price of American call option
that has strike price $k$ and expires in time $T$ from now
should be
\begin{align}
    \int_0^T \int_k^\infty p [X t = x] \cdot (x - k) \cdot e^{-r \cdot t} \cdot dx \cdot dt
\end{align}

The probability that the instrument price lies in the interval $X$
when the time is in the interval $T$ is
\begin{align}
    \int_T \int_X f x t \cdot dx \cdot dt
\end{align}
